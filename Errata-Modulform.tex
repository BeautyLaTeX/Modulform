%!TEX TS-program = xelatex
%!TEX encoding = UTF-8

% LaTeX source for the errata of the book ``模形式初步'' in Chinese
% Copyright 2020  李文威 (Wen-Wei Li).
% Permission is granted to copy, distribute and/or modify this
% document under the terms of the Creative Commons
% Attribution 4.0 International (CC BY 4.0)
% http://creativecommons.org/licenses/by/4.0/

% 《模形式初步》勘误表 / 李文威
% 使用自定义的文档类 AJerrata.cls. 自动载入 xeCJK.

\documentclass{AJerrata}

\usepackage{unicode-math}

\usepackage[unicode, colorlinks, psdextra, bookmarksnumbered,
	pdfpagelabels=true,
	pdfauthor={李文威 (Wen-Wei Li)},
	pdftitle={模形式初步勘误},
	pdfkeywords={}
]{hyperref}

\setmainfont[
	BoldFont={texgyretermes-bold.otf},
	ItalicFont={texgyretermes-italic.otf},
	BoldItalicFont={texgyretermes-bolditalic.otf},
	PunctuationSpace=2
]{texgyretermes-regular.otf}

\setsansfont[
	BoldFont=FiraSans-Bold.otf,
	ItalicFont=FiraSans-Italic.otf
]{FiraSans-Regular.otf}

\setCJKmainfont[
	BoldFont=Noto Serif CJK SC Bold
]{Noto Serif CJK SC}

\setCJKsansfont[
	BoldFont=Noto Sans CJK SC Bold
]{Noto Sans CJK SC}

\setCJKfamilyfont{emfont}[
	BoldFont=FandolHei-Regular.otf
]{FandolHei-Regular.otf}	% 强调用的字体

\renewcommand{\em}{\bfseries\CJKfamily{emfont}} % 强调

\setmathfont[
	Extension = .otf,
	math-style= TeX,
]{texgyretermes-math}

\usepackage{mathrsfs}
\usepackage{stmaryrd} \SetSymbolFont{stmry}{bold}{U}{stmry}{m}{n}	% 避免警告 (stmryd 不含粗体故)
% \usepackage{array}
% \usepackage{tikz-cd}  % 使用 TikZ 绘图
\usetikzlibrary{positioning, patterns, calc, matrix, shapes.arrows, shapes.symbols}

\usepackage{myarrows}				% 使用自定义的可伸缩箭头
\usepackage{mycommand}				% 引入自定义的惯用的命令

\newcommand{\bomega}{\symbf{\omega}}	% Boldface omega, for sheaves of differentials

\title{\bfseries 《模形式初步》勘误表}
\author{李文威}
\date{\today}

\begin{document}
	\maketitle
	以下页码和标号等信息参照科学出版社 2020 年 6 月出版之《模形式初步》, ISBN: 978-7-03-064531-9, 和网络版可能有异. 部分错误未见于网络版.
	
	\begin{Errata}
		\item[命题 1.1.9 证明最后一行]
		去掉``这''字.
		
		\item[(1.5.3)]
		\Orig 在 $\Gamma$ 作用下不变
		\Corr 在 $\gamma$ 作用下不变
		\Thx{感谢冯煜阳指正}
		
		\item[定义 1.6.7 第二项]
		\Orig $\delta' \Delta(x_0)$
		\Corr $\delta' D(x_0)$
		\Thx{感谢朱子阳指正}
		
		\item[定理 2.1.6 证明第一段结尾]
		\Orig ...... 给出 $\CC$ 上处处非零的全纯函数
		\Corr ...... 给出 $\CC$ 上的全纯函数, 在负整数处有一阶零点. 
		\Thx{感谢李时璋指正}

		\item[(2.5.4) 上两行]
		\Orig $J(-x, \tau) = J(x, \tau)$
		\Corr $J(-x, \tau) = -J(x, \tau)$
		\Thx{感谢冯煜阳指正}

        \item[定理 2.5.8 (iv) 最后一行]
        \Orig $\sigma^{\bar{v}}_r(n) := \cdots$
        \Corr $\sigma^{\bar{v}}_{k-1}(n) := \cdots$
        \Thx{感谢汤一鸣指正}
		
		\item[命题 3.5.6 的叙述和证明 (出现三次)]
		\Orig $\mathrm{Nrd}(q)^{-1} q$
		\Corr $\mathrm{Nrd}(q)^{-1} \overline{q}$ 
		\Thx{感谢李时璋指正}
		
		\item[命题 3.6.7 证明最后一段]
		\Orig 对 $u \in [0,x]$ 是一致的 ... 因为 $u \in [0,x]$
		\Corr 对 $u \in [0,y]$ 是一致的 ... 因为 $u \in [0,y]$
		\Thx{感谢李时璋指正}
		
		\item[命题 3.7.4 的前一段话 (纸本)]
		\Orig 内积系, 相对于
		\Corr 内积系相对于
		
		\item[注记 3.8.16]
		\Orig 对于全实域 $F$ 上仅对一个嵌入 $F \hookrightarrow \mathbb{R}$ 分裂的四元数代数 $B$
		\Corr 对于 $\mathbb{Q}$ 上对嵌入 $\mathbb{Q} \hookrightarrow \mathbb{R}$ 分裂, 但在 $\mathbb{Q}$ 上非分裂的四元数代数 $B$ 
		\Thx{感谢李时璋指正}
		
		\item[\S 4.4 第二段]
		``定义了模判别式...'' 之前 2.4 多出现了一次.
		\Thx{感谢汤一鸣指正}
		
		\item[练习 4.4.7 的表述]
		将列表第一项的 $M(1)_k$ 改为 $M_k(1)$.
		
		将最后一句``进一步, 说明 $S(1)$ 也来自一个分次理想 $S(1)_{\mathbb{Z}} \subset M(1)_{\mathbb{Z}}$.'' 改为: ``进一步描述 $M(1)_{\mathbb{Z}}$ 的分次理想 $M(1)_{\mathbb{Z}} \cap S(1)$.''
		\Thx{感谢李时璋指正}
		
		\item[练习 4.4.7 提示的第一句]
		\Orig 取 ...... $M(1)_{\mathbb{Z}} \cdot \Delta$
		\Corr 取 $M(1)_{\Z}$ 为所有 Fourier 系数均为整数的模形式给出的子环, 并应用前述定理.
		
		注: 相关的整性问题可以参考 Serge Lang 的 \textit{Introduction to Modular Forms} (Grundlehren der mathematischen Wissenschaften, Volume 222), Chapter X, Theorems 4.2---4.4. 论证是初等的.
		\Thx{感谢李时璋指正}
		
		\item[\S 4.5 第一句]
		应补上一句 ``本节的 Riemann 曲面默认紧.''
		\Thx{感谢李时璋指正}
		
		\item[定理 5.5.5 (i)]
		\Orig 则 $[\Gamma'_\lambda]$ 是中心元;
		\Corr 则对所有 $(h, k) \in \mathcal{D}$ 皆有 $[\Gamma'_{h,k}] \star [\Gamma'_\lambda] = [\Gamma'_{hd, kd}]$;
		\Thx{感谢于惠施指正}
		
		\item[定理 5.5.5 证明的第一条显示公式]
		\Orig $\displaystyle\bigsqcup_{a \in A}^n$
		\Corr $\displaystyle\bigsqcup_{a \in A}$
		
		\item[命题 5.5.7 证明中第三条显示公式末项]
		\Orig $\Z/hh'$
		\Corr $\Z/hh' \Z$
		\Thx{感谢朱子阳指正}

		\item[定理 6.2.5 (i)]
		\Orig 则 $[\Gamma'_\lambda(N)]$ 是中心元;
		\Corr 则对所有 $(h, k) \in \mathcal{D}(N)$ 皆有 $[\Gamma'_{h,k}(N)] \star [\Gamma'_\lambda(N)] = [\Gamma'_{hd, kd}(N)]$;

		\item[命题 6.3.2 之前]
		将``回忆到 \S 6.2 定义的子代数...''一句和后续的表格删除, 因为不正确而且不需要 (见下一条更正).
		\Thx{感谢李时璋指正}
		
		\item[命题 6.3.2 证明倒数第二段]
		\Orig 基于 $\EuScript{H}_1(N)$ 已知的结构...料理.
		\Corr 基于和引理 6.1.4 相同的论证, 说明 $\Gamma_1(N) \gamma\alpha\gamma^{-1} \Gamma_1(N) = \Gamma_1(N)\alpha\Gamma_1(N)$ 即可. 易见 $\gamma\alpha\gamma^{-1}$ 既属于 $\Delta_1(N)$, 又属于 $\alpha$ 的 $\Gamma_0(N)$-双陪集, 而命题 6.3.2 说明 \linebreak $\Gamma_1(N) \backslash \Delta_1(N) / \Gamma_1(N) \to \Gamma_0(N) \backslash \Delta_0(N) / \Gamma_0(N)$ 是双射, 于是 $\gamma\alpha\gamma^{-1}$ 和 $\alpha$ 确实属于相同的 $\Gamma_1(N)$-双陪集.
		\Thx{感谢李时璋指正}
		
		\item[\S 7.5 第一行 ``沿用...... 亦即 $a_0(f)=0$.'']
		删除此行.
		
		\item[练习 8.6.2 之前的显示公式]
		\Orig $\cdots \oplus \frac{1 + \sqrt{D}}{2}$
		\Corr $\cdots \oplus \Z \frac{1 + \sqrt{D}}{2}$
		
		\item[定理 8.6.4 的陈述]
		\Orig $[\cdot]: \End(E) \rightiso \mathcal{O}$
		\Corr $[\cdot]: \mathcal{O} \rightiso \End(E)$
		
		\item[定义 9.1.6 条列]
		将条列的两项修正为:
		\begin{itemize}
			\item $\Gamma(V, \bomega_\Gamma) := \mathcal{O}_V (\dd z \cdot \alpha^{-1}) |_{V \smallsetminus \{t\}}$, 其中 $V := \pi(U)$, 截面的限制映射按自明方式定义;
			\item $1 \mapsto \dd z \cdot \alpha^{-1}$ 给出平凡化 $\mathcal{O}_V \rightiso \bomega_\Gamma |_V$.
		\end{itemize}
		
		\item[引理 9.2.1]
		在引理陈述的最后, 亦即公式 (9.2.3) 之后补充一句 ``对 $\bomega^{\otimes (-1)}$ 的群作用是按 (9.1.4) 定义的.''
		\Thx{感谢李时璋指正}

		\item[注记 9.4.14 之上一句]
		\Orig 这是 Petersson 的...
		\Corr 这是 Petersson 内积的...
		
		\item[(10.1.1)] 将图表中的
		$\begin{tikzcd} \mathbb{C} \arrow[twoheadrightarrow, r, "\sim"] & \mathbb{C}^\times \end{tikzcd}$
		改成
		$\begin{tikzcd} \mathbb{C} \arrow[twoheadrightarrow, r] & \mathbb{C}^\times \end{tikzcd}$ .

		\item[定义 10.4.1]
		\Orig ... $\mathcal{W}_{\ell, p} \times \mathcal{W}_{\ell, p} \to \mathbb{Q}_\ell$, 满足 ...
		\Corr ... $\mathcal{W}_{\ell, p} \times \mathcal{W}_{\ell, p} \to \mathbb{Q}_\ell(-k-1)$, 其中 $\mathbb{Q}_\ell(-k-1)$ 是所谓的 Tate 挠 (仅影响 Galois 作用), 满足...
		
		\item[命题 10.5.5 (i)]
		将第二个 $\to$ 改成 $\rightiso$.
		
		\item[练习 10.6.5]
		删除提示.
		
		\item[注记 10.6.9]
		\Orig 故 $V_\ell(J)$ 为 $\Z_\ell$-模
		\Corr 故它们的 $\varprojlim_m$ 为 $\Z_\ell$-模
		
		另外, 将显示公式 $V_{f, \lambda} := V_\ell(J) \dotimes{\mathbb{T}_\ell, \phi_f} K_{f, \lambda}$ 中的 $\phi_f$ 改为 $\phi_{f, \lambda}$.
		
		\item[定理 10.6.10 后第二段, 从 ``回忆推论 6.5.6 和 6.5.7 ...'' 起]
		删除 ``回忆推论 6.5.6 和 6.5.7 ...'' 一段, 删除后续的命题 10.6.11 及其证明, 代换为 ``今后主要考虑 $f$ 为新形式的情形.'' 起新行, 接上原有的``我们以关于定理 10.6.7 的几点注记收尾.''
		
		\item[定义 10.7.2 之下两行]
		\Orig 同源等价
		\Corr 同源等价类.
		
		\item[练习 10.7.3 之后第二段: ``模性有一系列等价陈述...'']
		\Orig 无非是 Abel--Jacobi 映射 $\phi: X_0(N) \to J_0(N)$ 和 ...
		\Corr 无非是 Abel--Jacobi 映射 $\phi: X_0(N) \to J_0(N)$ (选定基点) 和 ...
		
		\item[参考文献 56] 该书已经正式出版 (Graduate Texts in Mathematics 288, Springer, 2021).
	\end{Errata}
\end{document}